% ---------------- Beginn Präambel --------------

% Datum: 06.11.2023
% Vorname: Daniel 
% Nachname: Bading
% Matrikelnummer: 405020
% E-Mail: daniel.bading@campus.tu-berlin.de

\documentclass{article}
\usepackage[utf8]{inputenc}
\usepackage[english]{babel}
\usepackage{blindtext}

% ----- TODO ------
% Aufgabe 1.1; letzter Stichpunkt: 
% 	Kennzeichnet den Beginn der Präambel und den Beginn der eigentlichen Text- Umgebung mit einem entsprechenden Kommentar.
% 

\title{Hausaufgabe 1 - Einführung in Latex}

\author{
  Bading, Daniel\\
  \texttt{daniel.bading@campus.tu-berlin.de}\\
  Matrikelnummer: 405020
  \and
  Mustermann, Max\\
  \texttt{max.mustermann@mustermann.de} \\
  Matrikelnummer: 432156
  }
 
\date{\today}
  
\begin{document}

\maketitle

\newpage

\tableofcontents

\newpage

% ------------ Beginn eigentlicher Text ---------

\section{Schrift in verschiedenen Schriftarten}

\subsection{fettgedruckt}

\begin{textbf}
\blindtext
\end{textbf}

\subsection{italic}

\begin{textit}
\blindtext
\end{textit}

\subsection{texttt}
\begin{texttt}
\blindtext  
\end{texttt}

\subsection{textsc}

\begin{textsc}
\blindtext
\end{textsc}

\newpage

\section{kombinierte Schriftarten}

jetzt schauen wir uns ein mal Texte mit kombinierten Schriftarten an: \\

\textbf{\textit{dies ist Beispiel 1 }} \\ 
\textbf{\underline{dies ist Beispiel 2 }} \\
\underline{\textsc{dies ist Beispiel 3 }} \\
\underline{\textit{dies ist Beispiel 4 }} \\

\newpage

\section{Schriften in verschiedenen Größen}

\subsection{Schrift in tiny}
\begin{tiny}
\blindtext
\end{tiny}

\subsubsection{Fett und tiny}

\begin{tiny}
\begin{textbf}
\blindtext
\end{textbf}
\end{tiny}

\subsection{Schrift in huge}

\begin{huge}
\blindtext
\end{huge}

\subsubsection{textsc und Huge}

\begin{huge}
\begin{textsc}
\blindtext
\end{textsc}
\end{huge}

\subsection{Schrift in Large}

\begin{Large}
\blindtext
\end{Large}

\subsubsection{italic und Large}

\begin{Large}
\begin{textit}
\blindtext
\end{textit}
\end{Large}

\subsection{Schrift in footnotesize}

\begin{footnotesize}
\blindtext
\end{footnotesize}

\subsubsection{texttt und footnotesize}

\begin{footnotesize}
\begin{texttt}
\blindtext
\end{texttt}
\end{footnotesize}

\end{document}