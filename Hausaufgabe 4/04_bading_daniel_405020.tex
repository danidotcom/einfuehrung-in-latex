% ---------------- Beginn Präambel --------------

% Datum: 06.11.2023
% Vorname: Daniel 
% Nachname: Bading
% Matrikelnummer: 405020
% E-Mail: daniel.bading@campus.tu-berlin.de

% ---------------- Ende Präambel --------------

\documentclass[11pt,a4paper]{scrartcl} % ändert das Layout des Dokuments
%\documentclass[a4paper]{article} % mit dieser Klasse funktioniert die Aufgabe 4.3 mit der split-Umgebung nicht! 

\usepackage[utf8]{inputenc}
\usepackage[english]{babel}
\usepackage{amsmath, amssymb}

\newcommand{\R}{\mathbb{R}}
\newcommand{\N}{\mathbb{N}}
\newcommand{\Z}{\mathbb{Z}}
\newcommand{\C}{\mathbb{C}}
\newcommand{\ph}{\varphi}
\newcommand{\p}{\partial}

\newcommand{\deff}[3]{#1 : #2 \to #3}
\newcommand{\dr}[1]{#1^{\ast}}


\title{Hausaufgabe 2 - Einführung in Latex}
\author{
  Bading, Daniel\\
  \texttt{daniel.bading@campus.tu-berlin.de}\\
  Matrikelnummer: 405020
  }
\date{\today}
  
\begin{document}

\maketitle

\newpage

\tableofcontents

\newpage

% ------------ Beginn eigentlicher Text ---------

\section{Hausaufgabe 4.1}

\begin{itemize}
\item Wir wissen, dass man den Vektorraum der komplexen Zahlen $\C$ als reellen Vektorraum $\R$ auffassen kann. Das Problem bei den natürlichen- und ganzen Zahlen ist, dass sowohl $\N$ als auch $\Z$ kein Vektorraum ist. 
\item Wir definieren uns einen Befehl, der es uns ermöglicht, Abbildungen wie $\deff{g}{\R}{\R^2}$ schneller und leichter zu definieren. Zusätzlich ist es sinnvoll sich auch einige Befehle zu definieren, um sich Schreibarbeit zu sparen - gerade wenn man zum Beispiel mit Dualräumen zutun hat. Dabei lässt sich dann nicht nur für jeden Raum separat der Dualraum definieren, sondern man fügt einfach einen Parameter ein. Anstatt zu schreiben: \textit{V\^{} \{\textbackslash ast\}} 
können wir einfach \textit{\textbackslash dr\{V \}} schreiben und erhalten: $\dr{V}$ als den Dualraum von V. 
\end{itemize}

\newpage

\section{Hausaufgabe 4.2}
%\subsection{Stetigkeit von $f \cdot g$}
\begin{itemize}
\item[(i)] Es gilt:
\begin{align*}
M_1 &= \{ ( x , y ) \in \R^2 \mid \vert x \vert + \vert y \vert = 1 \} \\
\overset{\circ} M_1 &= \emptyset \\
\partial M_1 &= \{ ( x , y ) \in \R^2 \mid \vert x \vert + \vert y \vert = 1 \} = M_1
\end{align*}
\item[(ii)] Mit Hilfe der Recheneregeln für Summen (Linearität) erhalten wir: 
\begin{align*}
\nabla (f + g)(x) &= 
\begin{pmatrix}
\partial_{x_1} (f(x) + g(x)) \\
\ldots \\
\partial_{x_n} (f(x) + g(x)) \\
\end{pmatrix} \\
&= \begin{pmatrix}
\partial_{x_1}f(x) + \partial_{x_1} g(x) \\
\ldots \\
\partial_{x_n}f(x) + \partial_{x_n} g(x)  \\
\end{pmatrix} \\
&= \nabla f(x) + \nabla g(x)
\end{align*}
\item[(iii)] Wir betrachten das folgende Gleichungssystem: 
\begin{align*}
\begin{pmatrix}
1 & 1 & 1 \\
2 & -2 & -1 \\
4 & 4 & 1 \\
\end{pmatrix}
\begin{pmatrix}
c_1 \\
c_2 \\
c_3 
\end{pmatrix}
= 
\begin{pmatrix}
2 \\
-1 \\
0
\end{pmatrix}
\end{align*}
Wir lösen es mit Hilfe des Gauss-Algorithmus:
\begin{align*}
\begin{matrix}
1 & 1 & 1 & \\
2 & -2 & -1 & \\
4 & 4 & 1 &
\end{matrix}
\begin{matrix}
\mid &  \\
\mid & \\
\mid &
\end{matrix}
\begin{matrix}
2 \\
-1 \\
0
\end{matrix}
\rightsquigarrow
\begin{matrix}
1 & 1 & 1 & \\
2 & -2 & -1 & \\
0 & 0 & -3 &
\end{matrix}
\begin{matrix}
\mid &  \\
\mid & \\
\mid &
\end{matrix}
\begin{matrix}
2 \\
-1 \\
-8
\end{matrix} \\
\rightsquigarrow
\begin{matrix}
1 & 1 & 1 & \\
2 & -2 & -1 & \\
0 & 0 & -3 &
\end{matrix}
\begin{matrix}
\mid &  \\
\mid & \\
\mid &
\end{matrix}
\begin{matrix}
2 \\
-1 \\
-8
\end{matrix} \\
\rightsquigarrow
\begin{matrix}
1 & 1 & 1 & \\
0 & -4 & -3 & \\
0 & 0 & -3 &
\end{matrix}
\begin{matrix}
\mid &  \\
\mid & \\
\mid &
\end{matrix}
\begin{matrix}
2 \\
-5 \\
-8
\end{matrix}
\end{align*}
\end{itemize}
\newpage
\section{Hausaufgabe 4.3}
\begin{itemize}
\item Die Funktion 
\begin{align*}
f : \R \to \R^2 , \quad f(x,y) = \begin{cases} 
x \sin(\frac{1}{y}) + y \cos(\frac{1}{x}) & \text{ falls } x \neq 0  \text{ und } y \neq 0 \\
0 & \text{ falls } x = 0 \text{ und } y = 0
\end{cases}
\end{align*}
ist stetig auf $\R^2$ außer auf $\{ ( x , y ) \in \R^2 \mid x = 0 $ oder $ y = 0 \} \backslash \{ ( 0 , 0 ) \}. $
\item Mit Hilfe des \texttt{\textbackslash cfrac}-Befehls setzen wir einen schönen Kettenbruch:
\begin{align*}
\cfrac{4}{\pi} = 1 + \cfrac{1^2}{3 + \cfrac{2^2}{5 + \cfrac{3^2}{7 + \cfrac{4^2}{9 + \cfrac{5^2}{\ddots}}}}}
\end{align*}
\item Tipp: \texttt{split}-Umgebungen: \\
Die folgende Rechnung bestätigt unsere Formel für den Laplace-Operator in Polarkoordinaten. 
\begin{align}
\begin{split}
\partial_r^2 g + \frac{1}{r} \cdot \partial_r g + \frac{1}{r^2} \cdot \partial_{\varphi}^2 g 
= &\cos(\varphi)^2 \partial_x^2 f + 2 \cos(\varphi) \sin(\varphi) \partial_x \partial_y f \\
&+ \sin(\ph)^2 \p_y^2 f + \frac{1}{r} ( \cos(\ph) \p_x f + \sin(\ph) \p_y f ) \\ 
&+ \frac{1}{r^2} ( -r ( \cos(\ph) \p_x f + \sin(\ph) \p_y f ) \\
&+ r^2 ( \sin(\ph)^2 \p_x^2 f + cos(\ph)^2 \p_y^2 f ) \\
&- 2 r^2 \sin(\ph) \cos(\ph) \p_x \p_y f ) \\
= & \p_x^2 f ( \cos(\ph)^2 + \sin(\ph)^2 ) \\
&+ \p_y^2 f ( \sin(\ph)^2 + \cos(\ph)^2 ) \\
&+ \p_x \p_y f ( 2 \cos(\ph) \sin(\ph) - 2 \sin(\ph) \cos(\ph) ) \\
&+ \p_x f ( \frac{1}{r} \cos(\ph) - \frac{1}{r} \cos(\ph) ) \\
&+ \p_y f ( \frac{1}{r} \sin(\ph) - \frac{1}{r} \sin(\ph) ) \\
= & \p_x^2 f + \p_y^2 f
\end{split}
\end{align}
\item Wir betrachten die Maxwellschen Gleichungen, wobei wir \textit{ \textbf{B, E, j} fett} setzen, statt sie mit Vektorpfeilen zu versehen, wie es in der Literatur auch vorkommt. Die Gleichungen sollen zentriert gesetzt werden, also nicht nach dem Gleichheitszeichen ausgerichtet werden.
\begin{gather*}
\nabla \cdot \textbf{E} = \frac{\rho}{\varepsilon_0} \\
\nabla \cdot \textbf{B} = 0 \\
\nabla \times \textbf{E} = - \frac{\p \textbf{B}}{\p t} \\
\nabla \times \textbf{B} = \mu_0 \textbf{j} + \mu_0 \varepsilon_0 \frac{\p \textbf{E}}{\p t} \\
\end{gather*}
\end{itemize}
\end{document}